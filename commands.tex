%%%%%%%%%%%%%%%%%%%%%%%%%%%%%%%%%%%%%%%%%%%%%%%%%%
% These are some new commands that may be useful 
% for paper writing in general. If other newcommands
% are needed for your specific paper, please feel 
% free to add here. 
%
% The currently available commands are organized in: 
% 1) Systems
% 2) Quantities
% 3) Energies and units
% 4) Detectors
% 5) particle species 
%%%%%%%%%%%%%%%%%%%%%%%%%%%%%%%%%%%%%%%%%%%%%%%%%%

% 1) SYSTEMS 
\newcommand{\pp}           {pp\xspace}
\newcommand{\ppbar}        {\mbox{$\text {p\overline{p}}$}\xspace}
\newcommand{\XeXe}         {\mbox{Xe--Xe}\xspace}
\newcommand{\PbPb}         {\mbox{Pb--Pb}\xspace}
\newcommand{\pA}           {\mbox{pA}\xspace}
\newcommand{\pPb}          {\mbox{p--Pb}\xspace}
\newcommand{\AuAu}         {\mbox{Au--Au}\xspace}
\newcommand{\dAu}          {\mbox{d--Au}\xspace}

% 2) QUANTITIES 
\newcommand{\s}            {\ensuremath{\sqrt{s}}\xspace}
\newcommand{\snn}          {\ensuremath{\sqrt{s_{\text{NN}}}}\xspace}
\newcommand{\pt}           {\ensuremath{p_{\text{T}}}\xspace}
\newcommand{\ptlambda}           {\ensuremath{p_{\text T}^{\Lambda}}\xspace}
\newcommand{\pttrig}           {\ensuremath{p_{\text{T}}^{\text{trig.}}}\xspace}
\newcommand{\ptassoc}           {\ensuremath{p_{\text{T}}^{\text{assoc.}}}\xspace}
\newcommand{\meanpt}       {$\langle p_{\text{T}}\rangle$\xspace}
\newcommand{\ycms}         {\ensuremath{y_{\text CMS}}\xspace}
\newcommand{\ylab}         {\ensuremath{y_{\text lab}}\xspace}
\newcommand{\etarange}[1]  {\mbox{$\left | \eta \right |~<~#1$}}
\newcommand{\yrange}[1]    {\mbox{$\left | y \right |~<~#1$}}
\newcommand{\dndy}         {\ensuremath{\text{d}N_\mathrmfamily{ch}/\mathrmfamily{d}y}\xspace}
\newcommand{\dndeta}       {\ensuremath{\text{d}N_\mathrmfamily{ch}/\mathrmfamily{d}\eta}\xspace}
\newcommand{\avdndeta}     {\ensuremath{\langle\dndeta\rangle}\xspace}
\newcommand{\dNdy}         {\ensuremath{\text{d}N_\mathrmfamily{ch}/\mathrmfamily{d}y}\xspace}
\newcommand{\Npart}        {\ensuremath{N_\text{part}}\xspace}
\newcommand{\Ncoll}        {\ensuremath{N_\text{coll}}\xspace}
\newcommand{\dEdx}         {\ensuremath{\texttext{d}E/\textrmfamily{d}x}\xspace}
\newcommand{\RpPb}         {\ensuremath{R_{\text pPb}}\xspace}
\newcommand{\zvtx}         {\ensuremath{Z_{\text vtx.}}\xspace}
\newcommand{\dphi}         {\ensuremath{\Delta\varphi}\xspace}

% 3) ENERGIES, UNITS
\newcommand{\nineH}        {$\sqrt{s}~=~0.9$~Te\kern-.1emV\xspace}
\newcommand{\seven}        {$\sqrt{s}~=~7$~Te\kern-.1emV\xspace}
\newcommand{\twoH}         {$\sqrt{s}~=~0.2$~Te\kern-.1emV\xspace}
\newcommand{\twosevensix}  {$\sqrt{s}~=~2.76$~Te\kern-.1emV\xspace}
\newcommand{\five}         {$\sqrt{s}~=~5.02$~Te\kern-.1emV\xspace}
\newcommand{\twosevensixnn}{$\sqrt{s_{\text{NN}}}~=~2.76$~Te\kern-.1emV\xspace}
\newcommand{\fivenn}       {$\sqrt{s_{\text{NN}}}~=~5.02$~Te\kern-.1emV\xspace}
\newcommand{\LT}           {L{\'e}vy-Tsallis\xspace}
\newcommand{\GeVc}         {Ge\kern-.1emV/$c$\xspace}
\newcommand{\MeVc}         {Me\kern-.1emV/$c$\xspace}
\newcommand{\TeV}          {Te\kern-.1emV\xspace}
\newcommand{\GeV}          {Ge\kern-.1emV\xspace}
\newcommand{\MeV}          {Me\kern-.1emV\xspace}
\newcommand{\GeVmass}      {Ge\kern-.2emV/$c^2$\xspace}
\newcommand{\MeVmass}      {Me\kern-.2emV/$c^2$\xspace}
\newcommand{\lumi}         {\ensuremath{\mathcal{L}}\xspace}

% 4) DETECTORS 
\newcommand{\ITS}          {\text{ITS}\xspace}
\newcommand{\TOF}          {\text{TOF}\xspace}
\newcommand{\ZDC}          {\text{ZDC}\xspace}
\newcommand{\ZDCs}         {\text{ZDCs}\xspace}
\newcommand{\ZNA}          {\text{ZNA}\xspace}
\newcommand{\ZNC}          {\text{ZNC}\xspace}
\newcommand{\SPD}          {\text{SPD}\xspace}
\newcommand{\SDD}          {\text{SDD}\xspace}
\newcommand{\SSD}          {\text{SSD}\xspace}
\newcommand{\TPC}          {\text{TPC}\xspace}
\newcommand{\TRD}          {\text{TRD}\xspace}
\newcommand{\VZERO}        {\text{V0}\xspace}
\newcommand{\VZEROA}       {\text{V0A}\xspace}
\newcommand{\VZEROC}       {\text{V0C}\xspace}
\newcommand{\Vdecay} 	   {\ensuremath{V^{0}}\xspace}

% 4) PARTICLE SPECIES 
\newcommand{\ee}           {\ensuremath{e^{+}e^{-}}} 
\newcommand{\pip}          {\ensuremath{\pi^{+}}\xspace}
\newcommand{\pim}          {\ensuremath{\pi^{-}}\xspace}
\newcommand{\kap}          {\ensuremath{\text{K}^{+}}\xspace}
\newcommand{\kam}          {\ensuremath{\text{K}^{-}}\xspace}
\newcommand{\pbar}         {\ensuremath{\text\overline{p}}\xspace}
\newcommand{\kzero}        {\ensuremath{{\text K}^{0}_{\rmfamily{S}}}\xspace}
\newcommand{\lmb}          {\ensuremath{\Lambda}\xspace}
\newcommand{\almb}         {\ensuremath{\overline{\Lambda}}\xspace}
\newcommand{\Om}           {\ensuremath{\Omega^-}\xspace}
\newcommand{\Mo}           {\ensuremath{\overline{\Omega}^+}\xspace}
\newcommand{\X}            {\ensuremath{\Xi^-}\xspace}
\newcommand{\Ix}           {\ensuremath{\overline{\Xi}^+}\xspace}
\newcommand{\Xis}          {\ensuremath{\Xi^{\pm}}\xspace}
\newcommand{\Oms}          {\ensuremath{\Omega^{\pm}}\xspace}
\newcommand{\degree}       {\ensuremath{^{\text o}}\xspace}