\chapter{Abstract}

Quantum chromodynamics (QCD) is the branch of fundamental particle physics that studies the strong interaction, which is responsible for binding quarks and gluons into the familiar protons and neutrons that compose almost all ordinary matter. One of the most exciting predictions of this theory is the existence of a new state of matter, known as the quark-gluon plasma (QGP). At extreme temperatures and densities, protons and neutrons dissolve into their constituent quarks and gluons, forming a soup-like plasma of deconfined partons. This QGP is thought to have existed in the early universe, thus studying its formation and properties can help answer questions about the evolution of the universe and the nature of the strong force. However, recreating the extreme conditions of the early universe can be achieved by smashing together heavy nuclei at \textit{very} high energies. This can only be done at the world's most powerful particle accelerators\footnote{Making the QGP the most expensive soup on the menu.}.

Unfortunately, QGP produced in these collisions only lasts for around $10^{-23}$ seconds, making it impossible to study the plasma directly. Instead, there are a few key experimental observables that are associated with the characteristics of this plasma which can be studied. One such signature of QGP formation is known as strangeness enhancement, where the production of strange quarks within the QGP is enhanced relative to the standard up and down quarks that compose protons and neutrons. Previously believed to be unique to heavy-ion collisions, recent measurements have indicated that this enhancement is also present in high multiplicity proton-proton (pp) and proton-lead (\pPb) systems as well. These measurements hint at the formation of a QGP in these smaller collision systems, calling into question the prediction that the formation of this plasma is only possible in heavy-ion collisions. While statistical and phenomenological models are capable of describing this enhancement in these smaller collision systems, the microscopic origins of strangeness enhancement are not well understood.

Jets, which are streams of hadrons produced by an initial hard scattering of the partons within the colliding nuclei, can be used to illuminate the underlying processes that produce these strange particles. By measuring the angular correlation between a high-momentum trigger hadron (as a proxy for a jet axis) and a lower momentum strange hadron, it is possible to differentiate the strangeness-producing mechanisms between hard (jet-like) and soft (underlying event) processes. This angular correlation technique can be used to study the production of strangeness as a function of multiplicity in these regimes, giving insight into the origins of the observed enhancement. 

This thesis presents the first results utilizing azimuthal angular correlations to measure the production of $\Lambda$ baryons---which are composed of an up, a down, and a strange quark ($uds$)---in \pPb collisions at \snn = 5.02 \TeV using the ALICE detector at the LHC. The correlation measurements are used to extract the $\Lambda$ jet-like and underlying event yields, as well as obtain the widths of the jets to provide more context for the observed jet production. These results are studied as a function of the \lmb momentum and collision multiplicity, which are used to quantify the enhancement of strange quark production in these different kinematic regimes. Comparisons with theoretical predictions are also made to provide a framework for interpreting the results of this thesis. Moreover, these yield measurements are compared with published measurements of the $\phi(1020)$ meson ($s\bar{s}$), which utilized similar techniques, to investigate the differences between open ($|S| > 1$) and hidden ($|S| = 0$) strangeness production. These strange measurements provide new insight into the production of strangeness in smaller collision systems, thus further constraining the microscopic origins of strangeness enhancement.