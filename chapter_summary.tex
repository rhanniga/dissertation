\chapter{Summary and outlook}

This thesis presents the first results using angular correlations to measure jet and non-jet $\Lambda$ and charged hadron production in p--Pb collisions at $\sqrt{s_{\text{NN}}} = 5.02$ TeV. Both the integrated per-trigger yields and the widths on the near- and away-side of the jet are extracted from the azimuthal correlation functions and studied as a function of associated momentum and event multiplicity. A strong dependence on multiplicity is observed for both the near- and away-side yields in the case of the $\Lambda$, whereas the charged hadron associated yields exhibit a much smaller (nearly statistically insignificant) dependence. The away-side yields also show a systematically larger increase with multiplicity than the near-side yields for both cases, hinting at modification of the away-side production due to jet-medium interactions. The h-$\Lambda$ and h-h near-side jet widths reveal a large dependence on \pt, becoming more collimated as momentum increases. The widths of the away-side jets are found to be independent of both \pt and multiplicity, however the larger systematic uncertainties introduce difficulties with excluding flat behavior. Comparing width values of the h-$\Lambda$ and h-h correlations, the h-$\Lambda$ near-side widths are found to be significantly ($> 2\sigma$) larger than the dihadron widths, whereas the away-side widths are consistent within uncertainties.

The per-trigger pair-wise yield ratios $R_{i}^{\Lambda/h}$ and $R_{i}^{\Lambda/\phi}$ ($i$ = near-side jet, away-side jet, UE) are also studied as a function of associated \pt and multiplicity. The $\Lambda/h$ ratios exhibit a clear ordering in each region for the entire multiplicity range in both \pt bins, with the UE ratios being larger than the away-side ratios, which are larger than the near-side ratios. The $\Lambda/h$ ratios in each region also reveal a strong dependence on multiplicity, with slopes that are greater than zero by nearly $5\sigma$ for both momentum bins. Furthermore, the away-side slopes are found to be systematically higher than the near-side slopes, indicating that the away-side $\Lambda$ production is more strongly enhanced than the near-side $\Lambda$ production with increasing multiplicity. The $\Lambda/\phi$ ratios in the near-side jet region are measured to be sytematically higher than both the away-side and UE ratios, hinting at a suppression of $\phi$ mesons along the jet axis. While the $\Lambda/\phi$ ratios show no significant dependence on multiplicity, the slopes of these ratios in the near- and away-side jet regions in the lower momentum bin show a small ($<2\sigma$) deviation from zero, indicating that the $\phi$ mesons may become less suppressed with increasing multiplicity at low \pt.

All measured observables are compared with predictions from the DPMJET model. The predicted near- and away-side yields are found to be in relatively good agreement with data in the dihadron case, but the h-$\Lambda$ yield predictions deviate from data by a large ($>40$\%) margin. DPMJET also fails to predict any of the observed multiplicity dependence for both the h-$\Lambda$ and h-h jet yields. However, the model is able to closely predict the near-side widths of the dihadron distribions across all multiplicity and momentum ranges, although it underpredicts both the h-$\Lambda$ near-side widths and the away-side widths for both ($\Lambda$, h) cases. The model also predicts a difference between the h-$\Lambda$ and h-h near-side widths, which is observed in data as well. The per-trigger $\Lambda/h$ and $\Lambda/\phi$ yield ratios  are consistently underpredicted by DPMJET, and exhbit no multiplicity dependence. Even still, DPMJET manages to predict the ordering of the $\Lambda/h$ ratios in each region (UE $>$ away-side jet $>$ near-side jet) and the enhancement of the jet-like $\Lambda/\phi$ ratio when compared to the UE region. 