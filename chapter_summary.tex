\chapter{Summary and outlook}

Quantum chromodynamics (QCD) is the theory that describes how the protons and neutrons, which compose almost all ordinary matter, are bound to one another within the nucleus of an atom. In this theory, the fundamental constituents of matter are \textit{not} protons and neutrons, but rather quarks and gluons. These quarks and gluons bind together through their color charge, creating the color neutral hadrons observed in nature. One of the most interesting predictions of QCD is that, at extreme temperatures and densities, these hadrons dissolve into their constituent quarks and gluons, forming a new state of matter known as the Quark-Gluon Plasma (QGP). Understanding the QGP, which is thought to be the state of the universe shortly after the Big Bang, can help answer questions about the nature of everyday matter and the evolution of the universe.

Studying the QGP requires recreating the extreme conditions of the early universe, which can only be acheived through high energy particle collisions. In such collisions, however, the QGP is only produced for a very short time, after which it expands and cools into the hadrons that are observed in the detector. Thus the QGP cannot be studied directly, and its formation and properties must be inferred from the information that is accessible to experiment. One of the first predicted experimental signatures for the formation of this plasma is the enhancement of strange hadron production, relative to non-strange pion production. While this enhancement was thought to be unique to heavy-ion collisions, experimental data from ALICE indicates that even high multiplicity pp and p--Pb collisions exhibit an onset of this enhanced production. This enhancement can only be described using statistical models, as its microscopic origin is not yet understood.

One way to study the strange particles that are produced in these high energy collisions is through jets, which are sprays of hadrons that come from a hard interaction of the partons inside the nuclei. By looking at how a high-momentum hadron (that represents a jet direction) and a lower momentum strange hadron are aligned in azimuthal angle, it is possible to separate the processes that create strangeness between hard (jet-related) and soft (QGP-related) ones. This angular correlation method can be used to further the understanding of how strangeness production depends on the multiplicity in these different regimes, thus providing insight to the onset of strangeness enhancement in smaller collision systems.

This thesis presents the first results using angular correlations to measure jet and non-jet $\Lambda$ and charged hadron production in p--Pb collisions at $\sqrt{s_{\text{NN}}} = 5.02$ TeV. By using the technique of two-particle angular correlations, the production of $\Lambda$ baryons can be separated into different kinematic regions: the near-side region (associated with jet-like strangeness production without QGP modification) away-side region (associated with jet-like production in the presence of the QGP), and the underlying event (tied to the uncorrelated strangeness production in the QGP). Both the yields and the jet widths on the near- and away-side regions are extracted from the azimuthal correlation functions and studied as a function of associated momentum and event multiplicity. A strong dependence on multiplicity is observed for both the near- and away-side yields in the case of the $\Lambda$, whereas the charged hadron associated yields exhibit a much smaller (nearly statistically insignificant) dependence. The away-side yields also show a systematically larger increase with multiplicity than the near-side yields for both cases, hinting at modification of the away-side production due to jet-QGP interactions. The h-$\Lambda$ and h-h near-side jet widths reveal a large dependence on \pt, becoming more collimated as momentum increases. The widths of the away-side jets are found to be independent of both \pt and multiplicity, however the larger systematic uncertainties introduce difficulties with excluding flat behavior. Comparing width values of the h-$\Lambda$ and h-h correlations, the h-$\Lambda$ near-side widths are found to be significantly ($> 2\sigma$) larger than the dihadron widths, whereas the away-side widths are consistent within uncertainties. This indicates that $\Lambda$ baryons are more readily produced in the peripheral regions of the jet cone, whereas charged hadrons are produced closer to the jet axis. This hints at a modification of the jet fragmentation process for strange hadrons, as more massive particles (like the \lmb) are expected to be produced closer to the jet axis.

The yield ratios $R_{i}^{\Lambda/h}$ and $R_{i}^{\Lambda/\phi}$ ($i$ = near-side jet, away-side jet, UE) are also studied as a function of associated \pt and multiplicity. The $\Lambda/h$ ratios exhibit a clear ordering in each region for the entire multiplicity range in both \pt bins, with the UE ratios being larger than the away-side ratios, which are larger than the near-side ratios. This indicates that relative \lmb production is larger in the UE (QGP) when compared to the jet-like regions. The $\Lambda/h$ ratios in each region also reveal a strong dependence on multiplicity, with slopes that are greater than zero by nearly $5\sigma$ for both momentum bins. This indicates that while the overall \lmb production is mostly concentrated in the UE, the observed enhancement of $\Lambda$ production with increasing multiplicity is \textit{also} driven by the jet-like regions. Furthermore, the away-side slopes are found to be systematically higher than the near-side slopes, indicating that the away-side $\Lambda$ production is more strongly enhanced than the near-side $\Lambda$ production with increasing multiplicity. Again, this suggests that the away-side jet strangeness production is modified by medium interactions. The $\Lambda/\phi$ ratios in the near-side jet region are measured to be sytematically higher than both the away-side and UE ratios, hinting at a suppression of $\phi$ mesons along the jet axis due to the lack of avaialble $s$-quarks in the unmodified jet. The slopes of these ratios in all kinematic regions are consistent with zero, indicating that the ratio is independent of collision centrality.

The measurements in this thesis are compared with theoretical predictions from the PHSD, EPOS and DPMJET models. PHSD is found to be in good agreement with all dihadron measurements, but fails to describe the overall $\Lambda$ yields. This is likely due to the requirement of a high momentum trigger hadron, which are not readily produced within the model. Even still, the shape of the near-side peak in the h-$\Lambda$ correlation distribution is well-described by PHSD. EPOS, on the otherhand, is able to describe the $\Lambda$ and hadron yields very well, but the correlation distributions are dominated by elliptic flow, making it impossible to extract the jet-like components. This flow contribution is much larger for the h-$\Lambda$ distributions, indicating that \lmb baryons in EPOS are mostly produced within the hydronamic core. The predicted near- and away-side yields from DPMJET are found to be in relatively good agreement with data in the dihadron case, but the h-$\Lambda$ yield predictions deviate from data by a large ($>40$\%) margin. DPMJET also fails to predict any of the observed multiplicity dependence for both the h-$\Lambda$ and h-h jet yields. However, the model is able to closely predict the near-side widths of the dihadron distributions across all multiplicity and momentum ranges, although it underpredicts both the h-$\Lambda$ near-side widths and the away-side widths for both ($\Lambda$, h) cases. The model also predicts a difference between the h-$\Lambda$ and h-h near-side widths, which is observed in data as well. This indicates that whatever process responsible for the production of strangeness in the periphery of the jet cone is contained within DPMJET. The per-trigger $\Lambda/h$ and $\Lambda/\phi$ yield ratios  are consistently underpredicted by DPMJET, and exhbit no multiplicity dependence. Even still, DPMJET manages to predict the ordering of the $\Lambda/h$ ratios in each region (UE $>$ away-side jet $>$ near-side jet) and the enhancement of the jet-like $\Lambda/\phi$ ratio when compared to the UE region. Thus the softer, uncorrelated processes implemented in DPMJET are responsible for the majority of the relative strangeness production. 


\section{Future outlook}

The measurements presented in this thesis strongly indicate that, while the dominating component for strangeness production comes from the QGP, the observed \textit{enhancement} of this production as a function of multiplicity has a large contribution from the jet-like regions. More still, the away-side jet component appears to undergo a larger enhancement than the near-side, hinting that the jet and medium are interacting in such a way that strangeness is more readily produced. These observations can be used to help fuel the theoretical models used to describe particle collisions, as such models are currently incapable of describing these results in their entirety. 

The techniques presented in this thesis can be easily extended to other collision systems, such as pp and Pb--Pb collisions, along with other particle species, such as the $K^0$. With the advent of the Run 3 data-taking period at the LHC, the ALICE detector will be able to collect more data than ever before, allowing for more precise measurements of these observables across a wider range of particles and collision systems. Such measurements will help further constrain the microscopic origins of this strange enhancement, thus providing more insight into the nature of the QGP and the universe as a whole.